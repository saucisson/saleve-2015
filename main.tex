\PassOptionsToPackage{svgnames}{xcolor}
\documentclass[english]{beamer}
\usepackage{polyglossia}
\usepackage{tikz}
\usepackage{minted}

\usetheme{Boadilla}
\usecolortheme{crane}

\usetikzlibrary{shapes}
\usetikzlibrary{plotmarks}
\usetikzlibrary{calc}
\tikzstyle{dd-root}=[
  shape=diamond
, minimum size=1em
, draw=black
, line width=1pt
, fill=red!10
]
\tikzstyle{dd-node}=[
  shape=circle
, minimum size=1.5em
, draw=black
, line width=1pt
, fill=yellow!10
]
\tikzstyle{dd-terminal}=[
  shape=rectangle
, minimum width=1.5em
, minimum height=1.5em
, draw=black
, line width=1pt
, fill=blue!10
]
\tikzstyle{dd-valuation}=[
  draw=none
, shape=rectangle
, minimum width=1em
, minimum height=1em
, midway
, sloped
]
\tikzstyle{dd-arc}=[
  -latex
, draw=black
, line width=1pt
]

\title[Decision Diagrams]{How to Compact and Already Compact Structure?}
\author[Alban Linard]{Didier Buchs, \emph{Alban Linard}, Emmanuel Paviot-Adet}
\date[7/12/2015]{7th December 2015}

\begin{document}

\maketitle

\begin{frame}
  \frametitle{Decision Diagrams?}
  \begin{itemize}
    \item Compact representation for \emph{sets} of data
    \item Used in model checking
    \item $10^{20}$ states encoded in~\cite{1992-burch-0}
    \item $\infty$ states encoded in~\cite{2009-htk-0}
  \end{itemize}

  \begin{block}{How?}
    \centering
    \begin{tikzpicture}
      \node [] at (1.5, 2.5) {\small firstname};
      \node [] at (4.5, 2.5) {\small age};
      \node [] at (7.5, 2.5) {\small lastname};
      \node [dd-node] (n1) at (0, 0  ) {};
      \node [dd-node] (n2) at (3, 1) {};
      \node [dd-node] (n3) at (3,-1) {};
      \node [dd-node] (n4) at (6, 1) {};
      \node [dd-node] (n5) at (6,-1) {};
      \node [dd-terminal] (t1) at (9, 0) {};
      \draw [dd-arc] (n1) -- (n2)
       node [dd-valuation, above] {Marie-Pierre};
      \draw [dd-arc] (n1) -- (n3)
       node [dd-valuation, below] {Catherine};
      \draw [dd-arc] (n2) -- (n4)
       node [dd-valuation, above] {67};
      \draw [dd-arc] (n3) -- (n4)
       node [dd-valuation, above] {67};
      \draw [dd-arc] (n3) -- (n5)
       node [dd-valuation, below] {38};
      \draw [dd-arc] (n4) -- (t1)
       node [dd-valuation, above] {Castel};
      \draw [dd-arc] (n5) -- (t1)
       node [dd-valuation, below] {Greiner};
    \end{tikzpicture}
  \end{block}
\end{frame}

\begin{frame}[fragile]
  \frametitle{From Data to Decision Diagram}
  \centering
  \begin{minted}[
    fontsize=\footnotesize
  ]{lua}
    { firstname = "Catherine"   , lastname = "Castel" , age = 67 }
    { firstname = "Marie-Pierre", lastname = "Castel" , age = 67 }
    { firstname = "Catherine"   , lastname = "Greiner", age = 38 }
  \end{minted}
  \begin{tikzpicture}
    \node [] at (1.5, 2.5) {\small firstname};
    \node [] at (4.5, 2.5) {\small age};
    \node [] at (7.5, 2.5) {\small lastname};
    \node [dd-node] (n1) at (0, 0  ) {};
    \node [dd-node] (n2) at (3, 1) {};
    \node [dd-node] (n3) at (3,-1) {};
    \node [dd-node] (n4) at (6, 1) {};
    \node [dd-node] (n5) at (6,-1) {};
    \node [dd-terminal] (t1) at (9, 0) {};
    \draw [dd-arc] (n1) -- (n2)
     node [dd-valuation, above] {Marie-Pierre};
    \draw [dd-arc] (n1) -- (n3)
     node [dd-valuation, below] {Catherine};
    \draw [dd-arc] (n2) -- (n4)
     node [dd-valuation, above] {67};
    \draw [dd-arc] (n3) -- (n4)
     node [dd-valuation, above] {67};
    \draw [dd-arc] (n3) -- (n5)
     node [dd-valuation, below] {38};
    \draw [dd-arc] (n4) -- (t1)
     node [dd-valuation, above] {Castel};
    \draw [dd-arc] (n5) -- (t1)
     node [dd-valuation, below] {Greiner};
  \end{tikzpicture}
\end{frame}

\begin{frame}
  \frametitle{Attributed Edges}
\end{frame}

\begin{frame}
  \frametitle{Edge-Valuation}
\end{frame}

\begin{frame}
  \frametitle{Can we do better}
\end{frame}

\begin{frame}
  \frametitle{Variables}
\end{frame}

\begin{frame}
  \frametitle{Terminals}
\end{frame}

\begin{frame}
  \frametitle{Stacks}
\end{frame}

\begin{frame}
  \frametitle{Edges}
\end{frame}

\begin{frame}
  \frametitle{Benchmarks}
\end{frame}

\begin{frame}
  \frametitle{Stack Compression}
\end{frame}

\begin{frame}
  \frametitle{Benchmarks}
\end{frame}

\begin{frame}
  \frametitle{Variable Order}
\end{frame}

\begin{frame}
  \frametitle{Benchmarks}
\end{frame}

\begin{frame}
  \frametitle{Conclusion}
\end{frame}

\begin{frame}[allowframebreaks]
  \bibliographystyle{alpha}
  \bibliography{decision-diagrams}
\end{frame}

\end{document}
